\documentclass[a4paper, 11pt, oneside]{report}

\usepackage[T1]{fontenc}    % fornisce la codifica adatta per il font della lingua italiana
\usepackage[utf8]{inputenc} % interpreta i caratteri immessi dall'editor, come i caratteri accentati italiani
\usepackage[italian]{babel} % convenzione per date, capitolo invece di chapter, regole di formattazione...

\usepackage{geometry}       % gestisce il layout del documento
% heightrounded modifica le regole di contenimento del testo per far rientrare il testo in un numero finito di righe
\geometry{a4paper, top=2cm, bottom=2cm, left=2.5cm, right=2.5cm, heightrounded}

\pagestyle{plain}           %numeri di pagina in fondo

\usepackage{hyperref}       %usato per collegamenti ipertestuali
\hypersetup{hidelinks}      %usato per rimuovere i riquadri dai link

% preambolo
\title{WeatherStyle\\Corso Fondamenti di Intellingenza Artificiale\\Prof. F.Palomba }
\author{Aurucci Raffaele\\Miglino Annalaura\\Palmieri Angelo\\Zammarrelli Francesco Giuseppe}
\date{16 gennaio 2023}

\begin{document}

    % sezione dedicata al preambolo
    \begin{titlepage}
        \maketitle
    \end{titlepage}

    % produzione dell'indice automatica
    \tableofcontents

    \part{Introduzione}
        \chapter{Cos'è WeatherStyle?}

            \section{Le motivazioni}
            Introduzione in cui si spiega il motivo per cui nasce il progetto

            \section{Gli obbiettivi}
            Obiettivi che si vogliono raggiungere

            \section{Specifiche PEAS}
            Performance Environment Actuators Sensors

            \subsection{Caratteristiche dell'ambiente}
            Caratteristiche dell'ambiente

    \part{Tecniche di risoluzione}
        \chapter{La ricerca locale}
        Brevissima spiegazione di cos'è la ricerca locale

            \section{Gli algoritmi genetici}
            Parlare brevemente della metaeuristica GA

            \section{Formulazione del problema}
            Formulazione del GA (codifica individui, operatori genetici - ATTENZIONE! sono stati usati diversi operatori per ogni algoritmo)
            \par \noindent Funzione di fitness (euristiche generali per il calcolo del punteggio)
            \par \noindent [potrebbe essere utile fare delle sottosezioni]

            \section{Vantaggi e svantaggi}
            Cosa ha senso e cosa no nella nostra risoluzione (vincoli, confronto ricerca esaustiva)
            \par \noindent [potrebbe essere utile fare delle sottosezioni]

            \section{Implementazione}
            Caratteristiche del framework e passi necessari alla risoluzione (eventuale pseduo codice)

        \chapter{Il machine learning}
        Brevissima spiegazione di cos'è il machine learning

            \section{CRISP-DM}
            Brevissima spiegazione di cos'è il modello CRISP-DM
            \par \noindent [potrebbe essere utile inserire un immagine]

            \section{Business Understanding}
            Obbiettivi di business, strumenti e tecnologie utilizzate

            \section{Data Understanding}
            Potrebbe essere utile inserire i link dei dataset trovati online e per ognuno costruire dei piccoli grafici
            (es: dataset maglie aveva dress, top e bottom e si mostra\ldots)
                \subsection{Dataset maglie e pantaloni}
                Questa fase può essere accorpata, magari si fa metà e metà
                \subsection{Dataset scarpe}
                Stesso discorso di cui sopra

            \section{Data Preparation}
            Qui parliamo del feature selection (feature che abbiamo selezionato), della fusione col dataset meteo e come
            è stata calcolata la variabile dipendente, eventuali imputazioni fatte\ldots
                \subsection{Dataset maglie e pantaloni}
                Come è stato costruito, strategia
                \subsection{Dataset scarpe}
                Come è stato costruito, strategia
                \subsection{Dataset meteo}
                Come è stato costruito, strategia
                \subsectin{La fusione dei dataset}
                Qui si parla di come è stata calcolata la variabile dipendente, facendo riferimento alla sezione dedicata
                alla formulazione del problema per il GA in particolare alla fitness

            \section{Modelling}
            Qui parliamo della tecnica di machine learning utilizzata, ovvero la regressione e dell'algoritmo utilizzato
                \subsection{La regressione}
                Brevissima spiegazione di che cos'è la regressione
                \subsection{L'albero di regressione}
                Spiegazione teorica di un albero di regressione
                \subsection{Ten folds cross-validation}
                Spiegazione teorica del 10 folds cross validation utilizzato
                \subsection{Valutazione delle metriche}
                Qui è possibile parlare delle metriche a cui ci siamo affidati, dei risultati ottenuti per ogni modello
                e inserire i grafici che mostrano la stima dell'errore rispetto alle predizioni fatte

            \section{Evaluation}
            Qui parliamo della sperimentazione empirica e dei risultati ottenuti

            \section{Deployment}
            Qui si parla leggermente dell'implementazione e delle scelte ingegneristiche per rendere il modello usabile
            quindi eventuali classi interfacce, ecc\ldots


            \part{Conclusioni e osservazioni}
            \chapter{Cosa scegliamo?}
                \section{Un confronto tra GA e ML}
                Quando conviene uno e quando conviene l'altro
                \section{Possiamo fare di meglio}
                Introduzione alle due osservazioni
                    \subsection{Usare il ML per il calcolo della funzione di fitness}
                    Spiegare cosa si può fare
                    \subsection{Usare un GA per costruire l'albero di regressione}
                    Spiegare cosa si può fare



\end{document}