\documentclass[a4paper, 11pt, oneside]{report}

\usepackage[T1]{fontenc}    % fornisce la codifica adatta per il font della lingua italiana
\usepackage[utf8]{inputenc} % interpreta i caratteri immessi dall'editor, come i caratteri accentati italiani
\usepackage[italian]{babel} % convenzione per date, capitolo invece di chapter, regole di formattazione...

\usepackage{geometry}       % gestisce il layout del documento
% heightrounded modifica le regole di contenimento del testo per far rientrare il testo in un numero finito di righe
\geometry{a4paper, top=2cm, bottom=2cm, left=2.5cm, right=2.5cm, heightrounded}

\pagestyle{plain}           % numeri di pagina in fondo

\usepackage{hyperref}       % usato per collegamenti ipertestuali
\usepackage{graphicx}       % usato per inserire immagini
\hypersetup{hidelinks}      % usato per rimuovere i riquadri dai link

% autostyle adatta lo stile delle citazioni alla lingua del documento,
% italian=gillments racchiude tra le virgolette caporali i campi che prevedono le virgolette
\usepackage[autostyle,italian=guillemets]{csquotes}
% usato per la generazione rif bibliografici, richiede l'uso di babel e csquotes, biblatex è il motore usato.
% Le citazioni sono definite in termini di etichette numeriche come [1],[2],...
\usepackage[bibstyle=numeric, citestyle=numeric-comp]{biblatex}

% preambolo
\title{\includegraphics[width=0.4\textwidth]{logo}\\WeatherStyle\\Corso Fondamenti di Intellingenza Artificiale\\Prof. F.Palomba}
\author{Repository github:\\\url{https://github.com/frankzamma/NC22_WeatherStyle_classe03.git}\\\\
        \\Aurucci Raffaele\\Miglino Annalaura\\Palmieri Angelo\\Zammarrelli Francesco Giuseppe}
\date{}

\begin{document}

    % sezione dedicata al preambolo
    \begin{titlepage}
        \maketitle
    \end{titlepage}

    % produzione dell'indice automatica
    \tableofcontents

    \part{Introduzione}
        \chapter{Cos'è WeatherStyle?}

            \section{Le motivazioni}
            Introduzione in cui si spiega il motivo per cui nasce il progetto

            \section{Gli obbiettivi}
            L'obbiettivo che Weather Style vuole raggiungere è quello di creare un agente intelligente che riesca a fornire
            all'utente dei suggerimenti sui capi d'abbigliamento che ritiene più adatti, questo basandosi su alcune informazioni
            metereologiche come la temperatura percepita, il clima e la stagione in cui è effettuata la previsione.

            \section{Specifiche PEAS}
            Performance Environment Actuators Sensors

            \subsection{Caratteristiche dell'ambiente}
                \begin{itemize}
                    \item \textbf{Tipo di ambiente:} guardaroba con informazioni meteorologiche.
                    \item \textbf{Completamente osservabile:} l'agente attraverso i sensori conosce lo stato
                    completo dell'ambiente in ogni momento, ovvero, ha visione completa dei capi presenti nel guardaroba
                    e delle informazioni meteorologiche.
                    \item \textbf{Deterministico:} lo stato successivo dell'ambiente è completamente determinato dallo
                    stato corrente e dall'azione eseguita dall'agente, dunque, è un ambiente che non cambia nel tempo.
                    \item \textbf{Episodico:} l'esperienza dell'agente si divide in episodi atomici, in ogni episodio
                    esegue una singola azione e non si lascia influenzare da ciò che è accaduto precedentemente.
                    \item \textbf{Statico:} l'ambiente non cambia nel mentre che l'agente sta eseguendo le proprie azioni.
                    \item \textbf{Discreto:} l'ambiente fornisce un numero limitato di percezioni\footnote{input percepiti
                    in un dato istante} e azioni distinte\footnote{una sola azione per ogni episodio}.
                    \item \textbf{Singolo agente:} l'ambiente consente la presenza di un singolo agente\footnote{nei
                    capitoli successivi vedremo tre agenti che agiranno singolarmente con uno scopo comune}
                \end{itemize}

    \part{Tecniche di risoluzione}
        \chapter{La ricerca locale}
        Gli algoritmi di \textbf{ricerca locale} a differenza degli algoritmi di ricerca tradizionale non hanno come scopo
        quello di trovare una sequenza di azioni che porti dallo stato iniziale allo stato obbiettivo, bensì lo stato
        obbiettivo rappresenta esso stesso la soluzione al problema, indipendentemente da come ci si è arrivati.
        \par \noindent Questo rende gli algoritmi di ricerca locale particolarmente adatti a problemi di configurazione
        e/o di ottimizzazione. Sono anche detti algoritmi di ``miglioramento iterativo'', poiché partono da un'ipotetica
        soluzione con lo scopo di migliorarla, secondo quella che viene chiamata \textbf{funzione obbiettivo}, una misura
        che serve all'algoritmo per comprendere se e quanto sta migliorando nelle iterazioni, condizione fondamentale per
        la corretta terminazione dell'algoritmo.
        \par \noindent In ultimo accenniamo al fatto che le soluzioni ottenute non sempre sono soluzioni ottimali, ma il
        più delle volte soluzioni sub-ottimali, ciò è dovuto dalla mancata esplorazione dell'intero spazio degli stati;
        compito del progettista è cercare di migliorare la capacità di esplorazione dell'algoritmo.

            \section{Gli algoritmi genetici}
            Parlare brevemente della metaeuristica GA
            Sebbene possa sembrare un argomento che si sia affacciato da poco sull'umanità, la storia degli algoritmi genetici
            ha inizio con il celeberrimo biologo inglese, Charles Darwin.
            \par \noindent Darwin pubblicò nel 1859 \textit{"L'origine della specie"}, libro che riscosse un enorme
            successo e nel quale si introduceva per la prima volta il concetto di \textbf{teoria dell'evoluzione} tramite
            un processo di \textbf{selezione naturale}.
            \par \noindent Negli anni '50 del XX secolo cominciò il diffondersi degli algoritmi evolutivi,
            i quali usano tutti i concetti formulati dalla teoria di Darwin, ovvero:
            \textit{genotipo}, \textit{fenotipo}, \textit{individuo}, \textit{popolazione}, \textit{evoluzione} e via dicendo.
            \par \noindent
            \\ \noindent Tra l'insieme degli algoritmi evolutivi troviamo anche gli \textbf{algoritmi genetici} che vennero
            presentati nel 1975 da John Holland, all'epoca detti \textit{genetic plans}.
            \par \noindent La definizione di algoritmi genetici è dunque la seguente:
            procedura di alto livello (meta-euristica) ispirata alla genetica per \textbf{definire}
            un algoritmo di ricerca.
            \par \noindent
            \\ \noindent Partendo da questa definizione si può dire che un algoritmo genetico evolve una \textbf{popolazione}
            di \textbf{individui} (le cosiddette soluzioni candidate) producendo un iterazione dopo l'altra delle
            soluzioni che migliorino sempre rispetto ad una \textbf{funzione obbiettivo}, fino a che non si è raggiunta
            la soluzione \textbf{ottimale} o si sia verificata una \textbf{condizione di terminazione}.
            Per poter creare nuove \textbf{generazioni} di individui bisogna applicare degli \textbf{operatori genetici},
            più nello specifico la \textbf{selezione}, il \textbf{crossover} e la \textbf{mutazione}.
            \par \noindent

            \section{Formulazione del problema}
            Formulazione del GA (codifica individui, operatori genetici)
            \par \noindent Funzione di fitness (euristiche generali per il calcolo del punteggio)
            \par \noindent [potrebbe essere utile fare delle sottosezioni]

            \section{Vantaggi e svantaggi}
            Cosa ha senso e cosa no nella nostra risoluzione (vincoli, confronto ricerca esaustiva)
            \par \noindent [potrebbe essere utile fare delle sottosezioni]

            \section{Implementazione}
            Per l'implementazione della soluzione basata su algoritmo genetico è stato utilizzato il framework
            \textbf{Jenetics} \cite{1}.

        \chapter{Il machine learning}
        Brevissima spiegazione di cos'è il machine learning

            \section{CRISP-DM}
            Brevissima spiegazione di cos'è il modello CRISP-DM
            \par \noindent [potrebbe essere utile inserire un immagine]

            \section{Business Understanding}
            Obbiettivi di business, strumenti e tecnologie utilizzate

            \section{Data Understanding}
            Potrebbe essere utile inserire i link dei dataset trovati online e per ognuno costruire dei piccoli grafici
            (es: dataset maglie aveva dress, top e bottom e si mostra\ldots)
                \subsection{Dataset maglie e pantaloni}
                Questa fase può essere accorpata, magari si fa metà e metà
                \subsection{Dataset scarpe}
                Stesso discorso di cui sopra

            \section{Data Preparation}
            Qui parliamo del feature selection (feature che abbiamo selezionato), della fusione col dataset meteo e come
            è stata calcolata la variabile dipendente, eventuali imputazioni fatte\ldots
                \subsection{Dataset maglie e pantaloni}
                Come è stato costruito, strategia
                \subsection{Dataset scarpe}
                Come è stato costruito, strategia
                \subsection{Dataset meteo}
                Come è stato costruito, strategia
                \subsection{La fusione dei dataset}
                Qui si parla di come è stata calcolata la variabile dipendente, facendo riferimento alla sezione dedicata
                alla formulazione del problema per il GA in particolare alla fitness

            \section{Modelling}
            Qui parliamo della tecnica di machine learning utilizzata, ovvero la regressione e dell'algoritmo utilizzato
                \subsection{La regressione}
                Brevissima spiegazione di che cos'è la regressione
                \subsection{L'albero di regressione}
                Spiegazione teorica di un albero di regressione
                \subsection{Ten folds cross-validation}
                Spiegazione teorica del 10 folds cross validation utilizzato
                \subsection{Valutazione delle metriche}
                Qui è possibile parlare delle metriche a cui ci siamo affidati, dei risultati ottenuti per ogni modello
                e inserire i grafici che mostrano la stima dell'errore rispetto alle predizioni fatte

            \section{Evaluation}
            Qui parliamo della sperimentazione empirica e dei risultati ottenuti
            Il risultato ottenuto non è da sottovalutare sicuramente per la non disponibilità dei dati.
            In particolare valutiamo

            \section{Deployment}
            Qui si parla leggermente dell'implementazione e delle scelte ingegneristiche per rendere il modello usabile
            quindi eventuali classi interfacce, ecc\ldots


    \part{Conclusioni e osservazioni}
        \chapter{Cosa scegliamo?}
            \section{Un confronto tra GA e ML}
                Quando conviene uno e quando conviene l'altro
            \section{Possiamo fare di meglio}
                Introduzione alle due osservazioni
                \subsection{Usare il ML per il calcolo della funzione di fitness}
                Spiegare cosa si può fare
                \subsection{Usare un GA per costruire l'albero di regressione}
                Spiegare cosa si può fare


    % sezione dedicata ai riferimenti bibliografici
    \begin{thebibliography}{9} % meno di 10, 99 meno di 100 e così via...
        \bibitem{1}
        Jenetics, Java Genetic Alghorithm Library,
        \url{https://jenetics.io/}.

        \bibitem{2}
        Weka, Waikato Environment for Knowledge Analysis,
        \url{https://www.cs.waikato.ac.nz/ml/weka/}.
    \end{thebibliography}

\end{document}